%   Filename    : chapter_1.tex 
    \section*{Chapter 1 }
\label{sec:researchdesc}    %labels help you reference sections of your document
\section{Introduction}
\subsection{Overview}
\label{subsec:overview}
    Philippine folk literature is the body of oral literature of the Filipino people. Folk literature typically undergoes classification into three categories: folk narratives, folk speech, and folk songs \cite{eugenio2007philippine}. Myths, legends, and folktales are included under the category of folk narratives, a form of literature that provides a narrative through prose or verse. These three forms of folk literature will serve as the focus of this study. Myths and legends are both regarded as truthful accounts of the past that provide explanations for the origins of entities in the environment. However, myths are often sacred and linked with religion, whereas legends tend to be secular in nature. On the other hand, folktales are fictitious prose narratives typically employed for entertainment purposes \cite{eugenio2007philippine}. In addition to their roles in explaining origins or providing entertainment, these three forms of folk literature often function as mediums for the communication of morals, traditions, and beliefs of the Filipino people. \shortciteA{eslit2023resilience} explored 10 popular folklores in the Philippines, examining their portrayal of Filipino culture and identity. Common themes in the analyzed folklore include environmental importance, respect for elders, and justice. These forms of folk literature have played significant roles in the conveyance and instillment of key values, traditions, and identity within particular ethnolinguistic groups. However, as \citeA{eugenio2007philippine} notes, there is a significant lack of collections of Philippine folk literature. Consequently, research on Philippine folk literature presents difficulties due to its wide dispersion across the country, the necessity for translations, and the rapid decline of this literary form, which limits available research. These problems were the focus of \citeA{eugenio2007philippine}’s work. However, due to the cost of the book and its dated nature, access to the resource has been limited. 

    According to \cite{dimzon2015}, there exists no digital ontology of Western Visayas folklore as digital ontology development was a new area of research. Their pioneering work serves as the start of the digitization of the Western Visayas folklore and is the basis of the researchers’ work. 
    With this, researchers propose the development of an ontology-based chatbot capable of answering questions and providing information about folk narratives, particularly those from Panay. 
    
     \shortciteA{jepsen2009just} offers a practical definition of ontology. Specifically, ontology as "a method of representing items of knowledge (ideas, facts, things—whatever) in a way that defines the relationships and classifications of concepts within a specified domain of knowledge." A chatbot is a software agent with the capability for engaging in human-like conversation. The researchers aim to provide the chatbot with knowledge and understanding of the relationships between concepts found in Panayanon folk narratives, which enables it to answer queries about them. Through the proposed system, the creation of a central hub of knowledge on Panayanon folk narratives facilitates the streamlining and accessibility of research and education on Panayanon folk narratives. Furthermore, the proposed system contributes to the preservation and promotion of cultural diversity and heritage, as globalization heightens the threat of the deterioration and disappearance of cultural heritage \cite{UNESCO_2001}. 

\subsection{Problem Statement}
\label{subsec:probstatement}
The body of knowledge regarding Philippine cultural heritage, specifically Philippine folk literature, remains limited. Despite efforts to collect and analyze this literature, the accessibility of such research is constrained by the cost of resources and the outdated nature of existing works. \citeA{eugenio2007philippine} affirms the lack of comprehensive collections and accessible resources on Philippine folk literature, resulting in significant challenges in the study, documentation, and promotion of this literary form.  

Damiana Eugenio, recognized as "Ina ng Folklor ng Pilipinas" by the U.P. Folklorists, Inc. and the U.P. Folklore Studies Program, has made significant contributions to the preservation of Philippine cultural heritage. Her book Philippine Folk Literature: An Anthology—the first volume in a seven-volume series—compiled over 150 texts and selections of proverbs and riddles from across the Philippines. However, due to the rapid digitization of global information and the fact that her works are now over 15 years old, their accessibility continues to diminish.

Recent efforts have sought to address this issue, with projects like the Aswang Project, created in 2006 by Jordan Clark. This project serves as an online resource for Philippine folklore, featuring articles about various myths, creatures, and spirits found throughout the country. Furthermore, in the terminal report of \citeA{dimzon2015}, they have collected and digitized Panayanon myths and legends by creating ontologies using Web Ontology Language (OWL). However, their work is not made publicly available and has not included folk tales from Panay; gaps remain in the collection of Panayanon folk narratives, which the researchers aim to explore further.

In the field of chatbots, \citeA{shawar2007chatbots} note that chatbots are designed to accommodate users' natural tendency to express their wishes through speaking, typing, or pointing \cite{zadrozny2000natural}. Consequently, chatbots present potential as educational tools, particularly as information retrieval systems. By offering quick and convenient responses similar to human interaction, chatbots hold promise for facilitating research and education. This potential is evidenced by the rapid growth of OpenAI’s ChatGPT, an artificial intelligence chatbot that gained one million users within days of its launch \cite{Mortensen_2024}.



\subsection{Research Objectives}
\label{subsecsec:researchobjectives}

\subsubsection{General Objective}
\label{subsec:generalobjective}

The researchers aim to add upon the ontology found in the pioneering work of \citeA{dimzon2015}. Subsequently, the researchers will develop a chatbot that is able to answer questions about Panayanon folk narratives. 

\subsubsection{Specific Objectives}
\label{subsec:specificobjectives}
Specifically, the researchers aim to:
\begin{enumerate}
    \item Add Panayanon folk tales to the ontology created by \citeA{dimzon2015}
    \item Add story details to the myths, legends, and folk tales in the expanded ontology
    \item Develop a chatbot capable of understanding English questions and responding with accurate and appropriate information from the expanded ontology

\end{enumerate}


\subsection{Scope and Limitations of the Research}
\label{sec:scopelimitations}

In this study, the development of a chatbot capable of understanding and responding to inquiries about Panayanon folk literature will be the primary focus. The scope of the folk literature analyzed will be limited to works originating from the island of Panay, specifically focusing on folk narratives, which include myths, legends, and folk tales. Additionally, the ontology used in the study will be based on the foundational work presented in the terminal report of \citeA{dimzon2015}, building upon and expanding these existing ontological frameworks to ensure comprehensive coverage of the key entities and relationships within Panayanon folk narratives. 

The native languages used in Panayanon folk narratives are the Panayanon languages: Hiligaynon, Akeanon, and Karay-a. However, the language used in the creation of the ontology and the chatbot will be English. This is to ensure that the chatbot can reach a wider audience, especially those who are more proficient in English, while still preserving and representing the cultural richness of the Panayanon folk narratives. 



\subsection{Significance of the Research}
\label{sec:significance}
The study holds significant value for the field of Panayanon cultural heritage and preservation for the following reasons:

The proposed system addresses the problem identified by \citeA{eugenio2007philippine} regarding the lack of published collections of Philippine folk literature. By serving as a central repository of knowledge for Panayanon folk narratives, the system is expected to facilitate easier access to Panayanon folk literature for researchers, students, educators, and the general public.

Additionally, the system seeks to address the issue of the decline of Panayanon oral literature by systematically collecting and digitizing these oral traditions, thereby contributing to their preservation for future generations.


