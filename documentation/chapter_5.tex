%   Filename    : chapter_5.tex 
\section*{Chapter 5}
\section{Conclusion}
    In this special problem project, the researchers built upon the foundational work of \cite{dimzonmyths} in the digital preservation of folk narratives from Panay Island. They enhanced the original ontology by introducing new classes and relationships that enable a richer and more detailed modeling of the stories' elements, thereby increasing its utility for complex scholarly analysis.
    
    The researchers successfully gathered a total of 137 Panayanon folk narratives, systematically encoding specific story elements into the expanded digital ontology based on the enhanced form. This effort contributed to the preservation of folk narratives by making them accessible in a structured, machine-readable, and queryable format. The resulting ontology serves as a foundational resource for future academic research in the humanities.

    To accomplish \citeA{manansala2007}) third and final pillar of ontology frameworks, the researchers developed a chatbot as an intuitive querying tool. Designed to be user-friendly, the chatbot enables users to retrieve the gathered data without requiring technical knowledge of ontology structures or query languages. The chatbot and the digital ontology were deployed through a website, ensuring that the repository of Panayanon folk narratives can be accessed. 

\subsection{Limitations}
     This study focused on a single anthology of Panayanon folk narratives, building upon the foundational work established by \citeA{dimzon2015}. While this approach ensured consistency with the existing ontology structure, it necessarily limited the scope of the data to a single source. 
     
     Furthermore, the developed chatbot was constructed as a prototype querying tool and remains constrained by the researchers' limited expertise in the domains of the humanities and literature. Particularly, they were challenged in understanding how to structure queries in a manner conventional to these fields.
    
    The deployment of the application was constrained by certain limitations, primarily due to resource restrictions associated with the use of the GCP free tier. In an effort to minimize costs, the researchers opted for this tier, which provides limited allocations of CPU and RAM. However, the resource requirements of the Rasa chatbot approached the bounds of these allocations, thereby posing challenges to the stability and scalability of the deployment.
    
\subsection{Recommendations}
    To enhance the breadth and richness of the digital ontology, future work should consider incorporating additional folk narratives documented by other researchers. Expanding the corpus in this way would not only diversify the representation of Panayanon traditions but also contribute more substantially to the preservation of West Visayan cultural heritage in digital form. Such efforts could serve as valuable references for future research in related fields.
    
    Moreover, extensive user testing involving students, faculty, and researchers specializing in literature and folklore is recommended. Insights from these user groups would provide critical feedback for refining the chatbot’s response accuracy, improving its handling of complex queries, and enhancing its overall utility as a querying tool.
    
    Finally, future iterations of the application would benefit from deployment under a paid cloud hosting tier. Access to expanded computational resources would enable greater reliability, support higher user load, and allow for more effective scalability.

\FloatBarrier