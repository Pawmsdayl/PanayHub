%   Filename    : chapter_4.tex 
\section*{Chapter 3}
\section{Research Methodology}
This chapter lists and discusses the specific steps and activities that will be performed  to accomplish the project. 

\subsection{Research Activities}

\subsubsection{Data Collection} 
    The researchers will collect Panayanon myths, legends, and folktales from reliable resource persons. Other sources may be explored, including written records, research papers, and digital archives. For validation, the collected folk narratives will be presented and consulted on by the researchers with literature experts from the UPV Division of Humanities and Center for West Visayan Studies to verify the authenticity of the collected folk narratives. 
    
    The expected outcome of this process is a comprehensive and authentic collection of folk narratives that reflects the breadth and richness of Panayanon culture. This step is scheduled to start in December 2024 and must be accomplished halfway through January 2025, with a total duration of one and a half (1.5) months.
    
\subsubsection{Ontology Enhancement} 
    The researchers will engage in extensive consultations with experts from the UPV Division of Humanities and Center for West Visayan Studies. They will focus on creating new classes for the digital ontology,  specifically story elements such as events and settings which are not present in the current ontology. Other possible classes may be explored. This will also be used to ensure consistency with standards in the field of literature.
    
    These new classes will be designed utilizing Protégé, an open-source ontology editor that supports OWL (Web Ontology Language) for formalizing domain knowledge. Each new class will be defined in terms of its relationships with other entities to create a structured and interconnected narrative representation. Protégé features such as logical constraints and reasoning will be utilized to ensure consistency and to infer relationships that enhance the semantic depth of the digital ontology.
    
    The expected outcome is an enhanced ontology structure that has more depth of information on Panayanon folk narratives than the original. This step is scheduled to start in mid-December 2024 and must be accomplished halfway through January 2025, with a total duration of one (1) month.

\subsubsection{Ontology Expansion}

    To follow good practices in the field of literature, the researchers will consult with experts to gain insights into the analysis of folk narratives, the identification of key story elements, and the contextual relationships between entities. With this, the researchers will closely read and examine each story from their collection, looking for relevant story elements and relationships. From their findings, they will expand the digital ontology by populating it with new stories, entities and relationships based on the enhanced ontological structure. 
    
    Protégé will be utilized for ontology expansion for its extensive support in OWL files and SPARQL querying, reasoning and consistency checking features, as well as collaboration features. Throughout this whole process, the researchers will present and consult with literature experts from the UPV Division of Humanities on the expanding ontology to validate the findings and ensure consistency with conventions and practices in the field of literature.
    
    The expected outcome is an expanded ontology that includes new details from the collected folk narratives based on the enhanced ontological structure. This step is scheduled to start in mid-January 2024 and must be accomplished by the end of April 2025, with a total duration of three and a half (3.5) months.

\subsubsection{Chatbot Development}
    In this step, the researchers will develop a chatbot prototype that can handle English queries from users, query the ontology to search for relevant data, and present the information to the user in comprehensible English sentences. Specifically, the researchers will utilize Python as the primary programming language, incorporating natural language processing (NLP) libraries such as SpaCy or NLTK to analyze and process user queries. Afterwards, the researchers aim to research and evaluate suitable chatbot models and methodologies, such as PAROT, and Sentence Transformers (SBERT) for entity-relationship extraction and intent identification. The researchers aim to research and evaluate suitable chatbot models and methodologies, such as PAROT,  for converting natural language queries into SPARQL queries and effectively interfacing with SPARQL and OWL files. The evaluation of the PAROT model by Ochieng (2020) involved the use of QALD-9 challenge metrics, including accuracy, recall, and F-measure. Similarly, the utilization of these metrics will be applied in the evaluation of the different models under study.

    The SPARQL queries will then be sent to an Apache Jena server where the OWL file of the ontology will be hosted. Finally, the query results will be formatted into English through NLP techniques. With each iteration of the chatbot, the researchers will perform tests to verify chatbot query accuracy and response relevance, assess user interaction with the chatbot, and measure response times and optimize as needed.

    With this chatbot, users will be able to interact with the ontology in natural language. This is in pursuit of data querying, which is \citeA{manansala2007}) third and final pillar of ontology frameworks. The prototype chatbot will only serve to demonstrate the feasibility of chatbots as an information retrieval tool of the digital ontology.

    The expected output is a chatbot prototype that can semantically understand complex user questions in English, and answer them with accurate information from the ontology in a natural language format. This step is scheduled to start in February 2024 and must be accomplished by the end of May 2025, with a total duration of four (4) months.

\subsubsection{Documentation}

    The researchers will document relevant results and information throughout the project. It shall cover  data, methodology, results, and analysis. Additionally, insights and validations provided by expert consultations and testing phases will also be documented. Google Docs will be used for its simplicity and familiarity with the researchers, and Overleaf will be utilized for final formatting. 
    
    Applying software engineering principles, the researchers will also create diagrams such as use case diagrams, and sequence diagrams. For diagrams, computer assisted software engineering (CASE) tools will be utilized. The software will also be documented and stored in a GitHub repository.
    
    This step ensures that all information has been transparently communicated for future reference to be used by other researchers and interested parties. The expected output is complete project documents, including technical details, the software itself, and a final project report. This step is scheduled to start in December 2024 and must be accomplished by the end of May 2025, with a total duration of five (6) months.

\subsection{Calendar of Activities}

Table \ref{tab:timetableactivities} shows a Gantt chart of the activities.  Each bullet represents approximately
one week worth of activity.

%
%  the following commands will be used for filling up the bullets in the Gantt chart
%
\newcommand{\weekone}{\textbullet}
\newcommand{\weektwo}{\textbullet \textbullet}
\newcommand{\weekthree}{\textbullet \textbullet \textbullet}
\newcommand{\weekfour}{\textbullet \textbullet \textbullet \textbullet}

%
%  alternative to bullet is a star 
%
\begin{comment}
   \newcommand{\weekone}{$\star$}
   \newcommand{\weektwo}{$\star \star$}
   \newcommand{\weekthree}{$\star \star \star$}
   \newcommand{\weekfour}{$\star \star \star \star$ }
\end{comment}

\begin{table}[ht]   %t means place on top, replace with b if you want to place at the bottom
\centering
\caption{Timetable of Activities} \vspace{0.25em}
\begin{tabular}{|p{2in}|c|c|c|c|c|c|c|c|} \hline
\centering Activities (2025)            & Dec & Jan   & Feb & Mar & Apr & May \\ \hline
Data Collection                         & \weekfour  & \weektwo ~~~ &  &  &  & \\ \hline
Ontology Enhancement                    & ~~~\weektwo  & \weektwo ~~~ & & & & \\ \hline
Ontology Expansion                      & & ~~~\weektwo & \weekfour & \weekfour & \weekfour & \\ \hline
Chatbot Development                     & & & \weekfour & \weekfour & \weekfour & \weekfour \\ \hline
Documentation                           & \weekfour  & \weekfour & \weekfour & \weekfour & \weekfour & \weekfour \\ \hline


\end{tabular}
\label{tab:timetableactivities}
\end{table}

